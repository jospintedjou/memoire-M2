
\newacronym{LPL}{LPL}{Ligne de produit logiciels}
\newacronym{ILP}{ILP}{Ingénierie de Ligne de Produits}
\newacronym{EMF}{EMF}{Eclipse Modeling Framework}
\newacronym{FODA}{FODA}{Feature Oriented Domain Analysis}
\newacronym{FORM}{FORM}{Feature Oriented Reused Method}
\newacronym{DSL}{DSL}{Domain Specific Language}

\newglossaryentry{LPL_def}
{
        name={LPL},
        description={un ensemble de logiciels partageant des propriétés communes, satisfaisant des besoins spécifiques pour un domaine donné et développés de façon contrôlée à partir d'un ensemble commun d’artefacts}
}
\newglossaryentry{domaine}
{
        name={domaine},
        description={un secteur d'activité ou une zone de connaissance dirigé par un ensemble de besoins et caractérisé par des concepts et des terminologies compréhensibles par les acteurs de ce secteur}
}
\newglossaryentry{ILP_def}
{
        name={ILP},
        description={un paradigme de développement de logiciels utilisant une plateforme logicielle commune et la personnalisation de masse}
}
\newglossaryentry{ingenierie_de_domaine}
{
        name={ingénierie},
        description={le processus de l'ingénierie de ligne de produits qui permet l'analyse, la spécification et l'implémentation des artéfacts logiciels dans un domaine qui sont utilisés pour construire les systèmes logiciels dans ce domaine}
}
\newglossaryentry{variabilite}
{
        name={variabilité},
        description={l'ensemble des différences, décrites de façon structuré, de toutes ou d'une partie des caractéristiques d'une famille de logiciels}
}
\newglossaryentry{feature}
{
        name={caractéristique},
        description={un aspect, une qualité ou une caractéristique, d'un logiciel ou d'une famille de logiciel, importante et visible par l'utilisateur}
}