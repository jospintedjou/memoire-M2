\usepackage[utf8]{inputenc}
\usepackage[T1]{fontenc}
\usepackage{lipsum}  %juste utile ici pour générer du faux texte}
\usepackage{amsmath,amsfonts,amssymb} %extensions de l'ams pour les mathématiques
%\usepackage{tikz}
\usepackage{graphicx} %pour insérer images et pdf entre autres
\usepackage[left=2.5cm,right=2.5cm,top=1.5cm,bottom=2cm]
{geometry} %réglages des marges du document selon vos préférences ou celles de votre établissement
\usepackage{shorttoc} % Pour un sommaire simplifié en début du mémoire
\usepackage{fancyhdr} %pour les en-têtes et pieds de pages
\setlength{ \headheight}{14.2pt} % hauteur de l'en-tête
%%%%%%%%%%%%%%%%%%%style front%%%%%%%%%%%%%%%%%%%%%%%%%%%%%%%%%%%%%%%%%
 \fancypagestyle{front}{ %
 \fancyhf{} %on vide l'en-tête
 \renewcommand{\headrulewidth}{0pt} %trait horizontal pour l'en-tête
 \renewcommand{\footrulewidth}{0.4pt} %trait horizontal pour le pied de page
 }
%\renewcommand{\chaptermark}[1]{%
%\markboth{\MakeUppercase{%
%\chaptername}\ \thechapter.%
%\ #1}{}}
%%%%%%%%%%%%%%%%%%%style main%%%%%%%%%%%%%%%%%%%%%%%%%%%%%%%%%%%%
 \fancypagestyle{plain}{ %
		\fancyhf{}
		\renewcommand{\chaptermark}[1]{\markboth{\chaptername\ \thechapter.\ ##1}{}} % redéfinition pour avoir ici les titres %des chapitres des sections en minuscules
		\renewcommand{\sectionmark}[1]{\markright{\thesection\ ##1}}
	 \fancyhead[C]{}
	 \fancyhead[L]{\rightmark} %
	 \fancyhead[R]{\leftmark}
	 %\fancyfoot[C]{}
	 \fancyfoot[R]{page \thepage} %
	%\fancyfoot[RO,LE]{page \thepage} %
	 %\fancyfoot[LO,RE]{Mon rapport}
	\fancyfoot[L]{Mémoire de Master II rédigé par Tedjou Jospin} %
}
%%%%%%%%%%%%%%%%%%%style back%%%%%%%%%%%%%%%%%%%%%%%%%%%%%%%%%%%%%%%%%
 \fancypagestyle{back}{ %
 \fancyhf{} %on vide l'en-tête
 \fancyfoot[C]{page \thepage} %
 \renewcommand{\headrulewidth}{0pt} %trait horizontal pour l'en-tête
 \renewcommand{\footrulewidth}{0.4pt} %trait horizontal pour le pied de pages
 }


\usepackage[english, french]{babel} %pour un document en français
\usepackage{listings} %pour insérer du code source
\usepackage{hyperref} %rend actif les liens, références croisées, toc…
\hypersetup{colorlinks, %
 citecolor=black, %
 filecolor=black, %
 linkcolor=black, %
 urlcolor=black}

%%%%%%%%%%%%%%%%%%%%%%%%%%%%biblio%%%%%%%%%%%%%%%%%%%%%%%%%%%%%%%%%%%%%
\usepackage[backend=biber]{biblatex}
%\bibliography{plain}
\addbibresource{../bibliographie/bibliographie.bib} % pour indiquer où se trouve notre .bib
\usepackage{csquotes} % pour la gestion des guillemets français.

%%%%%%%%%%%%%%%%%%%%%%%%%%%%%glossaire%%%%%%%%%%%%%%%%%%%%%%%%%%%%%%%%%%%
\usepackage{glossaries}
\makeglossaries

%%%%%%%%%%%%%%%%%%%%%%%%%%%%%index%%%%%%%%%%%%%%%%%%%%%%%%%%%%%%%%%%%
\usepackage{makeidx}
\makeindex

%%%%%%%%%%%%%%%%%%%%%%%%%%%%liste des abréviations%%%%%%%%%%%%%%
\usepackage{nomencl}
\makenomenclature
\renewcommand{\nomname}{Liste des abréviations, des sigles et des symboles}

%%%%%%%%%%%%%%%% environnement pour les résumés%%%%%%%%%%%%%%%%%%%%\makeatletter
\newenvironment{abstract}{ %
 \cleardoublepage
 %\null\vfil
 %\@beginparpenalty\@lowpenalty
 \begin{center} %
		\bfseries \abstractname
		%\@endparpenalty\@M
 \end{center}} %
 {\par\vfil\null}
\makeatother

